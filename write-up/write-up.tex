\documentclass[a4paper]{article}

\usepackage[english]{babel}
\usepackage[utf8]{inputenc}
\usepackage{amsmath}
\usepackage{graphicx}
\usepackage[colorinlistoftodos]{todonotes}

\title{RoadRunner}

\author{Pedro Cattori and Brando Miranda}

\begin{document}
\maketitle

\begin{abstract}
RoadRunner is a persistent, fault-tolerant, high-performance key-value store. 
RoadRunner avoids two round trip times by preparing everything in the Paxos log in advance from some sequence number to infinity. 
Our implementation of Multi-Paxos is independent of clock synchronization or time leases.
Our design of Multi-Paxos does not need to elect leaders explicitly by using Paxos instance to decide on an server id number.
Instead, from the servers that are alive, the one with the highest server id will consider himself the leader and prepare Paxos instances in advance (and eventually send accepts too).
The correctness of our Multi-Paxos design will not be affected if multiple servers think they are the leader. 
In fact, if the leader is unstable and the leader switches around constantly during the operation of RoadRunner, it will degrade gracefully into normal Paxos. 
Thus, in the worst-case where RoadRunner's servers die constantly, it will offer performance at least as good as normal Paxos.

\indent For persistence, Multi-Paxos will only reply to the proposers once it has persistence the acceptors on disk. This is to not change already decided values. 
RoadRunner also writes the key-value store to disk and persists the local min, so that it knows what operations from the Multi-Paxos log to not re-apply. 
Furthermore, if one server has its disk contents crash, we have a mechanism to ensure correctness of the whole service when we add back a functional server to the system.
\end{abstract}

\section{Protocol Design}

Conceptually, the reason for this is because the effective leader will send the highest epoch round along with any prepares or accepts he does, so two things will happen; first, old leader will have a lower epoch round so servers will reject old accepts, second, new leader will always prepare before sending accepts, so they will learn of old decided value if there are any. 


\subsection{Overview}

\subsubsection{Overview of Multi-Paxos}

In the case of a stable leader Multi-Paxos will have a performance benefit by avoid round trip times to start Paxos instances in the log.
Multi-Paxos will prepare all sequences in the Paxos log from a given sequence number to infinity with the current round number.
However, what used to be a round number in normal Paxos will have different semantic in our Multi-Paxos.
In our design, a round number will still be used to reject old accepts and prepare's, however, they will also correspond to the epoch round of the current leader. 
Therefore, round numbers will be referred to as epoch numbers and the current acting leader will have the highest epoch round number (corresponding to the round in which he is the leader).
Therefore, the leader will send prepare and accept messages with his epoch number and, if any other old leader tries to send accept or prepare messages with his lower epoch number, servers that have already been prepared with the current leader will reject and inform him that he should find out who the real leader is.
Therefore, we ensure that old leader cannot change any already decide values because if a majority of the new servers have been prepared by the new leader, they will reject any old accept or prepare message and inform the confused leader about the new leader.
Furthermore, new leader cannot possibly change old decided values either. 
This is because when a new leader is formed, he will *always* send prepare messages before sending accept messages.
Therefore, whenever he tries propose any new values, he will always be informed by some overlapping majority of a decided value by a previous epoch number, if one exists.
Thus, new leader cannot change decision because they prepare before they accept.


\subsubsection{Overview of the whole Service and Persistance}



\end{document}